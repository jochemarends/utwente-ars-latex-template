\documentclass{paper}

\addbibresource{./references.bib}

\title{Put your title here}
\author{Koen Blanke, Jochem Arends, Milan Lopuhaä-Zwakenberg}
\track{CS and IST} % Computer Science

\begin{document}

\maketitle

\begin{abstract}

Type the Abstract here (with no ‘Abstract’ heading, 150 words max.).
This document contains instructions for authors who intend to write a paper for the ARS Conference at the University of Twente.
It provides information on the layout and arrangement of the publication as well as on e.g. the typefaces that are used.
For all relevant elements of the publication, a description is given.
Authors may use any text editor, but they should eventually deliver the paper in Portable Document Format (PDF).
The final manuscript should be submitted via
Canvas, using the file name that is requested there.
The deadline for sending the final manuscript can also be found on Canvas.
In case of questions, please contact the conference secretariat (w.dankers@utwente.nl).

\end{abstract}

\section{This is a 1st order heading}

This is the paper template for the Academic Research Skills conference proceedings.
Please ensure that you follow this format exactly, so that the ARS team can ensure that the layout of all publications is consistent.
The maximum page count for this publication is 6 pages.
The minimum page count is 4 pages.
The page size is ISO A4.
The template uses two columns, each 89,3 mm.
wide, with a gutter of 5,5 mm.
The default typeface is Cambria, with a font size of 9 points.
The line spacing is set to single.
In the publication, the language is British English.
Ensure that the language options for spell-check are set accordingly.

\subsection{This is a 2nd order heading}

This section describes how to specify the title, names of authors, track and date of the publication.
The title is typed at the position indicated, in Cambria, 14 points.
The title does not use capitalisation, except for the first letter, for standard abbreviations and for names.
Titles do not incorporate signs like copyrights, trademarks etc.

Two lines below the title (at the position indicated), the name of the authors is typed in 11 point typeface.
For each name, please type the first name, then any middle initials and then your last name.
Composite last names (‘van den ...’ or ‘de ...’) should not be abbreviated and should be written in speaking order.
All names and initials are capitalised – except for prefixes in composite names.
As an example, a name could be John F.H. Doe or Jan de Ontwerper.
Do not use academic titles.

The pre-master track that you are enrolled in is specified two lines below the name of the author, again, at the position provided.
This template already contains all possible options, so please just remove the part of the line provided that does not apply.
The next line states the submission date, i.e. the date at which the document is submitted via Canvas.
Provide the date as month dd, yyyy.
An example could be: January 1, 2030.

\subsection{Abstract and keywords}

Start typing the abstract immediately below the horizontal line.
The abstract should provide a brief summary of the contents of your publication.
It can not contain any references.
Do not start the abstract with a heading ‘Abstract’.
The abstract should not consist of more than 150 words.
Below the abstract leave one empty line and then enter the keywords of your publication.
Do not start this line with a heading ‘Keyword’.
You are free to select appropriate keywords, with a maximum of five.
Select meaningful keywords, and avoid generic words like ‘design’.
Also, do not use company/organisation names as keywords.

\subsection{Headings}

The publication template uses two levels of headings; Heading 1 and Heading 2 are provided in the Styles in this word document.
To avoid any complexities with the numbering of headings, the numbering is done manually.
Do not use a third level heading that is numbered.
Usually, third level headings should/can be avoided.
The first order heading is set in a bold typeface, with a line spacing of 1,5.
The space above and below the heading is 12 and 3 points respectively.
The first line is indented 0,25 mm.
Numbering is in the format: “n.”, with a space after the number.
The second order heading is set in italic, with single line spacing.
The space above and below the heading is 9 and 12 points respectively.
The first line is indented 0,25 mm.
Numbering is in the format: “n.m.” with a space after the number.

\section{Body text}

The main text of the publication is set in Cambria 9 pts, with justified text in the column.
The line spacing is single, without any space above and below the paragraph.
The first line has an indentation of 0,25 cm.
Do not use empty lines between paragraphs of text.
Ensure that all tables, figures and schemes are cited in the text in numerical order.
Trade names should have an initial capital letter, and trademark protection should be acknowledged in the standard fashion, using the superscripted characters for trademarks and registered trademarks respectively.
All measurements and data should be given in SI units where possible, or other internationally accepted units.
Abbreviations should be used consistently throughout the text, and all nonstandard abbreviations should be defined on first usage.

\subsection{Style}

The following list summarizes several important points of style to keep in mind when preparing your publication:

\begin{itemize}
    \item Use bold for emphasis, but keep its use to a minimum.
    \item Avoid using underlining in your paper
    \item Use a consistent spelling style throughout the paper (UK)
    \item Use single quotes
    \item Use \%, not ‘percent’
    \item Do not use ampersands (\&) except as part of the official name of an organisation, company or pre-master track.
    \item Keep hyphenation to a minimum. Do not hyphenate 'coordinate' or 'non' words, such as 'nonlinear'
    \item Do not end headings with full stops
    \item Do not start headings at the foot of a column or with only one line of text below; put the heading on the next column or page
    \item Leave one character space after all punctuation
    \item Use lists as shown in this example; you can use the style ‘List Paragraph’ as provided.
\end{itemize}

\section{Figures and tables}

Try to include your tables and figures in your text close to the position where you reference the figure or table.
Preferably, include the figure or table immediately after (or before) the current paragraph – not dividing the paragraph.
Photos, graphs, line drawings, and other nontabular graphics should be high resolution (at least 200 dpi).
Any text in a figure should be at least 8 points in size.

\begin{figure}[H]
    \centering
    \includegraphics[width=\columnwidth]{example-image-a}
    \caption This is a test
\end{figure}

\cite{watanabe2019simd}

\end{document}

